\documentclass[11pt]{article}

\usepackage{alltt,fullpage,graphics,color,epsfig,amsmath, amssymb}
\usepackage{hyperref}
\usepackage{boxedminipage}
\usepackage[ruled,vlined]{algorithm2e}

\newcommand{\floor}[1]{\lfloor #1 \rfloor}
\newcommand{\ceil}[1]{\lceil #1 \rceil}

\title{CS 510 Experimental Lab}
\author{Daniel Campos}
\date{April 14th,2021}
\begin{document}
\maketitle
\end{document}
\section{General Info}
Our team is blender, our team is: Daniel Campos and our task is the 2021 CLEF eHealth IR Task on consumer health search: Subtask 1
Team Name & Team Members}
\section{Goal of Participation}
The goal of participation in this task is to build real world experience on how to create a high quality search stack on a novel corpus. 
In this I seek 

3) Sec 1. Goal of Participation:  Describe your goal of participating in the task. For example, you may include a discussion about what you were hoping to learn by participating in the chosen task or the question(s) that you hope to investigate via the experiments. (A few sentences are sufficient.)
\section{Activities}
4) Sec 2. Activities:  Give a detailed description of the activities that your team have done. For example, we first tried XX and found it didn't work. We then tried YY, and found that ..... Or, we compared two methods, A and B. We optimized method A by ...., and optimized B by ..... Or, we focused on studying the effectiveness of method X and varies multiple parameters of the method including ....  (Write at least 0.5 page for this part.) 
\section{Findings}

5) Sec. 3 Findings: Give a description of the findings that you have made via your experiments. This is similar to the part of a research paper on analysis of experiment results. Make sure to include at least one table or plot that shows the performance numbers of different methods or the same method with different parameter settings. Explain what the results tell us in terms of answering your research question. Do the results make sense to you? Try to provide an explanation of the results if you can. For example, why do you think method A performs better than method B? Or why do you think method A works better when the parameter is set to a smaller value? Also, try to look at the performance in detail by examining the performance on each single test instance. For example, in a retrieval experiment, the performance is often reported as an average over a set of different queries. Looking into the performance on each individual query can often reveal interesting findings that we can't easily see by only looking at the average performance.  (Write at least 0.5 page)
\section{Error Analysis}
To explore the failure of our model we looked at a query where our system has the lowest performance:{} 
In this query we find that 
6) Sec. 4. Error analysis: Try to look at at least one specific instance where a method has failed to see if you can find out any particular reason for the failure. Error analysis is a great way to gain interesting new insights about how to improve an existing method. (Try to write at least a few sentences about this, but no more than 0.5 page.) 
\section{Summary}
In this experimental lab what we have learned how a performant search engine can be setup on a novel search task. As shown in our results, we found that varing 
7) Sec. 5. Summery:  Briefly describe what you've learned from the Experimental Lab. (A few sentences would be sufficient.) 
\section{Task division}
Since Daniel is the only member of this group he has done everything. 